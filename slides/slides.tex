\documentclass{beamer}
\usepackage{listings}

\title[Effprog]%(optional, only for long titles)
{185.190 Effiziente Programme}
\subtitle{Aufgabe: Hash-Tabelle}
\author{Berger G., Hotz-Behofsits C., Reisinger M., Schmidleithner T.}
\date{WS12/13}
\subject{Informatik}

\lstset{breakatwhitespace,
language=C,
keywordstyle=\color{blue},
stringstyle=\color{red},
commentstyle=\color{gray},
columns=fullflexible,
keepspaces,
breaklines,
tabsize=3, 
showstringspaces=false,
extendedchars=true}

\newcommand{\success}[1]{\textcolor{green}{#1}}
\newcommand{\fail}[1]{\textcolor{red}{#1}}

\begin{document}
	% Inlining
	\defverbatim[colored]\sInlining{%
\begin{lstlisting}[tabsize=8,basicstyle=\ttfamily]
inline unsigned long hash(char *addr, size_t len);
inline void insert(char *keyaddr, size_t keylen, int value);
inline int lookup(char *keyaddr, size_t keylen);
\end{lstlisting}}

	% Lineares Sondieren
	\defverbatim[colored]\sLinear{%
\begin{lstlisting}[tabsize=8,basicstyle=\ttfamily\footnotesize]
void insert(char *keyaddr, size_t keylen, int value) {
    struct hashnode **l;
    int startPosition = hash(keyaddr, keylen) & (HASHSIZE-1);
    int position = startPosition;
    do {
        l = &ht[position];
        position = (position + 1) % HASHSIZE;
    } while(*l != NULL && position != startPosition);

    if (*l == NULL) {
        struct hashnode *n = malloc(sizeof(struct hashnode));
        n->keyaddr = keyaddr;
        n->keylen = keylen;
        n->value = value;
        *l = n;
    }
}
\end{lstlisting}}

	% Quadratisches Sondieren
\defverbatim[colored]\sQuadratOne{%
\begin{lstlisting}[tabsize=8,basicstyle=\ttfamily\footnotesize]	
void insert(char *keyaddr, size_t keylen, int value) {
    struct hashnode **l;
    int startPosition = hash(keyaddr, keylen) & (HASHSIZE-1);
    int position = startPosition; int i = 0;
    do {
        l = &ht[position];
        position = (startPosition + (int) pow(-1, i) + (i*i/2)) % HASHSIZE;
        i++;
    } while(*l != NULL && position != startPosition);

    if (*l == NULL) {
        struct hashnode *n = malloc(sizeof(struct hashnode));
        n->keyaddr = keyaddr;
        n->keylen = keylen;
        n->value = value;
        *l = n;
    }
}
\end{lstlisting}}	

\defverbatim[colored]\sQuadratTwo{%
\begin{lstlisting}[tabsize=8,basicstyle=\ttfamily\footnotesize]	
int lookup(char *keyaddr, size_t keylen) {
    int startPosition = hash(keyaddr, keylen) & (HASHSIZE-1);
    int position = startPosition;
    struct hashnode *l;
    int i = 0;
    do {
        l = ht[position];
        if (l == NULL) {
            break;
        }
        if (keylen == l->keylen && memcmp(keyaddr, l->keyaddr, keylen) == 0) {
            return l->value;
        }
        position = (startPosition + (int) pow(-1, i) + (i*i/2)) % HASHSIZE;
        i++;
    } while(position != startPosition);
    return -1;
}
\end{lstlisting}}	

	% Packed
	\defverbatim[colored]\sPacked{%
\begin{lstlisting}[tabsize=8,basicstyle=\ttfamily]
struct hashnode {
  char *keyaddr;
  size_t keylen;
  int value;
} __attribute__((__packed__));
\end{lstlisting}}

	% Schritt 4
	\defverbatim[colored]\sFourOne{%
\begin{lstlisting}[tabsize=4,basicstyle=\ttfamily]
struct hashnode *next; /* link ext. chaining */
\end{lstlisting}}

\defverbatim[colored]\sFourTwo{%
\begin{lstlisting}[tabsize=4,basicstyle=\ttfamily\footnotesize]
int position = hash(keyaddr, keylen) & (HASHSIZE-1);
struct hashnode *l; l = ht[position];
while (l != NULL) {
	if (keylen == l->keylen &&
		memcmp(keyaddr, l->keyaddr, keylen) == 0)
		return l->value;
	if (position < HASHSIZE - 1)
		l = ht[++position];
	else
		break;
}	
\end{lstlisting}}	

	\begin{frame}
	\titlepage
	\end{frame}

  \begin{frame}
    \frametitle{Ausgangssituation}
		
	\begin{itemize}
		\item Testaufruf:
		\begin{itemize}
			\item gcc -lm hash.c -o hash
      \item perf stat -e cycles,cache-misses,branch-misses,instructions ./hash input input2
     \end{itemize}
		\item Ergebnis:
		\begin{itemize}
			\item Cycles: 6,156,600,783\\
			\item Instructions:  1,939,017,297\\
			\item Cache-misses:  37,721,251\\
			\item Branch mispredictions: 18,758,092\\
		\end{itemize}		
		\item Testrechner:
		\begin{itemize}
			\item Intel Core i5-2520M CPU @ 2.50GHz
			\item Cache-size:
			\begin{itemize}
				\item Lvl 3: 3072 KB
				\item Lvl 2: 512 KB
				\item Lvl 1: 128 KB
			\end{itemize}
			\item RAM: 4GB DDR-3
		\end{itemize}
	\end{itemize}
	\end{frame}
  
  \begin{frame}
  	\frametitle{Schritt 1}
  	gcc -O3 -lm hash.c -o hash\\[1em]
  	\textbf{Vorher:}
  	\begin{itemize}
			\item Cycles: 6,156,600,783\\
			\item Instructions: 1,939,017,297\\
			\item Cache-misses: 37,721,251\\
			\item Branch mispredictions: 18,758,092\\
		\end{itemize}	
		
		\textbf{Nachher:}
  	\begin{itemize}
			\item Cycles: 3,705,108,800 (\success{+ 39,82})\\
			\item Instructions: 1,158,823,277 (\success{+ 40,24})\\
			\item Cache-misses: 37,394,499 (\success{+ 0,87})\\
			\item Branch mispredictions: 20,203,186 (\fail{- 7,70})\\
		\end{itemize}	
  \end{frame}
  
  \begin{frame}
  	\frametitle{Schritt 2}
		\framesubtitle{Inlining}
  	\textbf{Vorher:}
  	\begin{itemize}
			\item Cycles: 3,705,108,800 \\
			\item Instructions: 1,158,823,277\\
			\item Cache-misses: 37,394,499\\
			\item Branch mispredictions: 20,203,186\\
		\end{itemize}	
				
		\textbf{Nachher:}
  	\begin{itemize}
			\item Cycles: 3,995,922,639 (\fail{- 7,85})\\
			\item Instructions:  1,158,154,470 (\success{+ 0,06})\\
			\item Cache-misses: 37,389,502 (\success{+ 0,01})\\
			\item Branch mispredictions: 20,691,809 (\fail{- 2,42})\\
		\end{itemize}			
		Keine Verbesserung $\Rightarrow$ \fail{entfernt}.
  \end{frame}
  
    \begin{frame}
  	\frametitle{Code: Schritt 2}
		\framesubtitle{Inlining}
		\sInlining	
  \end{frame}
  
  
  \begin{frame}
  	\frametitle{Schritt 3}
  	\framesubtitle{Packed}
  	\textbf{Vorher:}
  	\begin{itemize}
			\item Cycles: 3,705,108,800 \\
			\item Instructions: 1,158,823,277\\
			\item Cache-misses: 37,394,499\\
			\item Branch mispredictions: 20,203,186\\
		\end{itemize}	
		
		\textbf{Nachher:}
		\begin{itemize}
			\item Cycles: 3,760,116,819 (\fail{- 1,48})\\
			\item Instructions: 1,158,688,286 (\success{+ 0,01})\\
			\item Cache-misses: 37,372,930 (\success{+ 0,06})\\
			\item Branch mispredictions: 19,799,458 (\success{+ 2,0})\\
		\end{itemize}	
		Verschlechterung $\Rightarrow$ \fail{entfernt}.
  \end{frame}
  
    \begin{frame}
  	\frametitle{Code: Schritt 3}
  	\framesubtitle{Packed}
		\sPacked
  \end{frame}
    
  \begin{frame}
  	\frametitle{Schritt 4}
  	\framesubtitle{Lineares Sondieren}
  	\textbf{Vorher:}
  	\begin{itemize}
			\item Cycles: 3,705,108,800 \\
			\item Instructions: 1,158,823,277\\
			\item Cache-misses: 37,394,499\\
			\item Branch mispredictions: 20,203,186\\
		\end{itemize}	
		
		\textbf{Nachher:}
		\begin{itemize}
			\item Cycles: 4,588,844,030 (\fail{- 23,85})\\
			\item Instructions: 1,315,414,647 (\fail{- 13,51})\\
			\item Cache-misses: 58,530,839 (\fail{- 56,52})\\
			\item Branch mispredictions: 25,859,851 (\fail{- 28,00})\\
		\end{itemize}	
		Verschlechterung $\Rightarrow$ \fail{entfernt}.
  \end{frame}
  
  \begin{frame}
  	\frametitle{Code: Schritt 4}
  	\framesubtitle{Lineares Sondieren}
		\sLinear
  \end{frame}
    
  \begin{frame}
  	\frametitle{Schritt 5}
  	\framesubtitle{Quadratisches Sondieren}
  	\textbf{Vorher:}
		\begin{itemize}
			\item Cycles: 3,705,108,800 \\
			\item Instructions: 1,158,823,277\\
			\item Cache-misses: 37,394,499\\
			\item Branch mispredictions: 20,203,186\\
		\end{itemize}	
				
		\textbf{Nachher:}
		\begin{itemize}
			\item Cycles: 5,948,874,039 (\fail{- 60,56})\\
			\item Instructions: 2,588,119,362 (\fail{- 123,34})\\
			\item Cache-misses: 43,166,841 (\fail{- 15,44})\\
			\item Branch mispredictions: 22,792,713 (\fail{- 12,82})\\
		\end{itemize}	
		Verschlechterung $\Rightarrow$ \fail{entfernt}.
  \end{frame}
  
  \begin{frame}
  	\frametitle{Code: Schritt 5 (1/2)}
  	\framesubtitle{Quadratisches Sondieren}
		\sQuadratOne
  \end{frame}
  
  \begin{frame}
  	\frametitle{Code: Schritt 5 (2/2)}
  	\framesubtitle{Quadratisches Sondieren}
		\sQuadratTwo
  \end{frame}
      
\end{document}
